\documentclass[conference]{IEEEtran}
\IEEEoverridecommandlockouts
% The preceding line is only needed to identify funding in the first footnote. If that is unneeded, please comment it out.
\usepackage{cite}
\usepackage{amsmath,amssymb,amsfonts}
\usepackage{algorithmic}
\usepackage{graphicx}
\usepackage{textcomp}
\usepackage{xcolor}
\usepackage[shortlabels]{enumitem}
\def\BibTeX{{\rm B\kern-.05em{\sc i\kern-.025em b}\kern-.08em
    T\kern-.1667em\lower.7ex\hbox{E}\kern-.125emX}}
\begin{document}

\title{Evaluating the Value of Professional Body Computer Science Degree Accreditation in the UK}

% {\footnotesize \textsuperscript{*}Note: Sub-titles are not captured in Xplore and
% should not be used}
% \thanks{Identify applicable funding agency here. If none, delete
% this.}

\author{\IEEEauthorblockN{Tom Crick\IEEEauthorrefmark{1}, Alastair
    Irons\IEEEauthorrefmark{2}, Liz Bacon\IEEEauthorrefmark{3}, Kevin
    Chalmers\IEEEauthorrefmark{4}, James
    H. Davenport\IEEEauthorrefmark{5}, Paul
    Hanna\IEEEauthorrefmark{6}, }Alan Hayes\IEEEauthorrefmark{5},
  Cathryn Knight\IEEEauthorrefmark{1}, Steve
  Pettifer\IEEEauthorrefmark{7} and Tom Prickett\IEEEauthorrefmark{8}
\IEEEauthorblockA{\IEEEauthorrefmark{1}Swansea University, Swansea, UK; Email:
  \{thomas.crick,cathryn.knight\}@swansea.ac.uk}
\IEEEauthorblockA{\IEEEauthorrefmark{2}Sunderland University, Sunderland, UK; Email:
alastair.irons@sunderland.ac.uk}
\IEEEauthorblockA{\IEEEauthorrefmark{3}Abertay University, Dundee, UK; Email: l.bacon@abertay.ac.uk}
\IEEEauthorblockA{\IEEEauthorrefmark{4}University of Roehampton, London, UK;
Email:kevin.chalmers@roehampton.ac.uk}
\IEEEauthorblockA{\IEEEauthorrefmark{5}University of Bath, Bath, UK; Email: \{j.h.davenport,a.hayes\}@bath.ac.uk}
\IEEEauthorblockA{\IEEEauthorrefmark{6}Ulster University, Newtownabbey, UK; Email: jrp.hanna@ulster.ac.uk}
\IEEEauthorblockA{\IEEEauthorrefmark{7}University of Manchester, Manchester, UK; Email:
  steve.pettifer@manchester.ac.uk}
\IEEEauthorblockA{\IEEEauthorrefmark{8}Northumbria University, Newcastle upon Tyne, UK; Email:
  tom.prickett@northumbria.ac.uk}}

% \author{\IEEEauthorblockN{1\textsuperscript{st} Given Name Surname}
% \IEEEauthorblockA{\textit{dept. name of organization (of Aff.)} \\
% \textit{name of organization (of Aff.)}\\
% City, Country \\
% email address or ORCID}
% \and
% \IEEEauthorblockN{2\textsuperscript{nd} Given Name Surname}
% \IEEEauthorblockA{\textit{dept. name of organization (of Aff.)} \\
% \textit{name of organization (of Aff.)}\\
% City, Country \\
% email address or ORCID}
% \and
% \IEEEauthorblockN{3\textsuperscript{rd} Given Name Surname}
% \IEEEauthorblockA{\textit{dept. name of organization (of Aff.)} \\
% \textit{name of organization (of Aff.)}\\
% City, Country \\
% email address or ORCID}
% \and
% \IEEEauthorblockN{4\textsuperscript{th} Given Name Surname}
% \IEEEauthorblockA{\textit{dept. name of organization (of Aff.)} \\
% \textit{name of organization (of Aff.)}\\
% City, Country \\
% email address or ORCID}
% \and
% \IEEEauthorblockN{5\textsuperscript{th} Given Name Surname}
% \IEEEauthorblockA{\textit{dept. name of organization (of Aff.)} \\
% \textit{name of organization (of Aff.)}\\
% City, Country \\
% email address or ORCID}
% \and
% \IEEEauthorblockN{6\textsuperscript{th} Given Name Surname}
% \IEEEauthorblockA{\textit{dept. name of organization (of Aff.)} \\
% \textit{name of organization (of Aff.)}\\
% City, Country \\
% email address or ORCID}
% }

\maketitle

\begin{abstract}
This Research-to-Practice Full Paper critically evaluates the
perceived value offered by the computer science degree accreditation
scheme of a large professional body/learned society based in the
UK. The value and relevancy of degree accreditation in computing and
engineering-related disciplines remains contested in many
jurisdictions, especially in the context of graduate employability and
higher education institutions needing to ``better meet the needs of
industry and society''. The degree accreditation process has been
criticised for being overly burdensome and bureaucratic; adversarial
rather than enhancement- led; simultaneously too hard and too easy to
obtain; perceived as focused on generating revenue for the accrediting
body rather than providing value to the discipline; and for being
colonial and paternalistic if the accreditor is not from the same
jurisdiction as the institution being accredited. Equally, the value
presented by degree accreditation has been cited as a kitemarking
exercise, promoting international graduate mobility and a globally
portable workforce; highlighting internationally recognised standards;
focusing on outcomes and raising output standards; promoting and
disseminating best practice; ensuring industry relevance of curricula;
and even reigniting the debate on the need for a “licence to practise”
in certain areas of IT and software engineering. However, it is
unclear which of these various perceptions dominates in term of
stakeholder, from across academia, professional body/learned society,
industry, government, the wider public and indeed learners themselves.

However, taking into consideration the potentially divergent
requirements and perceptions of these various stakeholders, it is the
academic institution that chooses to (or to not to) seek accreditation
of the degree programmes they offer; as such, this research study
primarily focuses on the perceptions and experience of academic
faculty. It draws from across the career ladder and academic hierarchy
in the UK, from early-career junior faculty, through to tenured
professors and senior leadership. Furthermore, this takes place in the
context of evolving international accreditation processes, such as
AHEP4 and EQANIE.

This study thus addresses two key research questions in the context of
UK computer science degree accreditation:\\

\begin{enumerate}[i)]
\item What are perceived to be the key aspects of value presented by
  degree accreditation processes?
\item What are perceived as the key challenges for future degree accreditation processes.
\end{enumerate}

This work adopts a mixed methods approach, drawing on the quantitative
and qualitative findings from a national-level survey in 2021 of
computer science faculty in the UK, collected through a convenience
sampling approach. It is augmented by in-depth focus groups and
semi-structured interviews with representative UK faculty identified
through purposive sampling. The outcomes of this wider study provides
deeper insights into the current perceived value of computer science
degree accreditation in the UK, as well as making a number of
recommendations across policy and practice for how it may change in
the near future. In doing so, it presents a rationale for improved
articulation and communication of the benefits of degree accreditation
(in both the UK and internationally), as well as action to address the
challenges for both professional bodies and memorandum organisations
to ensure they maintain their relevance in the context of ongoing
global disruption from COVID-19 to learners, institutions, employers
and society as a whole.
\end{abstract}

\begin{IEEEkeywords}
TBC
\end{IEEEkeywords}

\section{Introduction}
The assurance of quality through degree accreditation by Professional,
Statutory and Regulatory Bodies (PSRBs) is very much a feature of
higher education in the UK and other jurisdictions for Computing, Engineering and a wide range of other disciplines.  Accreditation regimes need to continue to evolve in order to address the  dynamic and emerging expectations of the UK educational, economic and policy environment and thereby maximise the value they provide to higher education institutions, as well as to
industry and society as a whole.

Accreditation by PSRBs is not without criticism. The processes are
unnecessarily bureaucratic and constrain innovation~\cite{Harvey2004};
 there are dangers of accreditation streams being revenues streams
in their own right rather than for the benefit of a discipline or
wider society~\cite{Knight_2015}; and the regimes are or colonial
and paternalistic in nature~\cite{Mutereko2018}. However, the value of professional body degree
accreditation regimes as a kite-marking exercise and to support
a globally-portable and recognised workforce remains high~\cite{Knight_2015}. 

In the UK, the predominant form of accreditation for computer science
and cognate disciplines is through two professional bodies: BCS, The
Chartered Institute for IT (BCS) and the Institution of Engineering
and Technology (IET). The BCS accredits mainly to three professional registrations Chartered Engineer (CEng), Chartered IT Professional (CITP) and Registered IT Technician (RITTech). Two international initiatives, the Washington
Accord~\cite[for CEng]{Washington2022} and Seoul Accord~\cite[for
CITP]{Seoul2022} underpin the accreditation provided. These memoranda promote consistency and parity in computer science education globally and hence the internation relevance of the curriculum. RITTech is used for accreditation work experience i.e. placements and other similar activities that are integrated into degree programmes. As part of accreditation, the BCS and IET broadly check two things~\cite{BCS2020, IET2022}:
\begin{enumerate}
			%\item Is the quality of experience provided by a HEI appropriate for accreditation?
			\item Are the exit standards of the programme appropriate for the
			level of accreditation sought?  A number of standards are considered
			including entry, progression, retention, awards and graduate
			employment.  This is supplemented by other evidence of the quality
			of the provision, for example external examiners reports, the most
			recent subject review, annual review information, evidence of employer
			involvement, departmental underpinning research. Together this evidences that a
			programme is of an appropriate quality to support accreditation.
			\item Are the curricula exit standards of a programme consistent
			with the learning outcomes expected for the accreditation sought? The
			expected exit standards should conform with the international
			memorandum (Washington or Seoul Accord or both).
\end{enumerate}

In the UK, for one professional body, namely The BCS, The Chartered Institute for IT (BCS) a project has been in progress since 2019 to explore and enhance the value proposed by degree accreditation. The first step in this project was to define what the intended value of the accreditation regime was by internal workshops. This resulted in a statement of the value ~\cite{crick-et-al-accred:cep2020,itnowaccred:2020} provided by accreditation was 
\begin{itemize}
	\item Raising output standards
	\item Promoting internationally-recognised standards
	\item Ensuring curricula relevance
	\item Disseminating good practice
	\item Industry relevance
	\item Accrediting work experience
\end{itemize}
This was alongside implementing a process for continuous improvement whilst striving to reduce the administrative overhead and hence perceived bureaucracy involved.

Building on the findings of UK Government’s 2016 Shadbolt review ~\cite{shadbolt2016shadbolt}, and the above work towards defining a statement of value, the next step in the project was to assesses whether computer science degree programmes in the UK needs to address a
refreshed set of accreditation criteria. To this end, and to assure a broad range of views were included, a steering group of practitioners, industrialists, academics and representatives from other UK engineering/technology professional bodies was established under an independent chair in summer 2020. The preliminary tasks completed by the steering group and a wider supporting team included: a review of progress in reforming academic accreditation of computing related degree courses since the Shadbolt report was published in 2016; an evaluation of what currently works well and is valued by the various stakeholders and establish if fundamental changes are required (and if so, what are they); and gathering recommendations for reforming accreditation to fulfil the purpose of validating that graduates have gained sufficient academic knowledge, understanding and competencies for a successful professional career ~\cite{irons-et-alposter:sigcse2021}. The initial work focused upon focus group discussions. The views of the wider sector where also desired to be included within the enhancement project. This leads to the focus of this paper which reports and draws upon the quantitative and qualitative findings from a national-level survey in 2021 of computer science faculty in the UK, collected through a convenience
sampling approach.

\section{Research questions}

Two research questions are considered as part of this study:\\

\begin{description}
	
	\item [RQ1:] What are perceived to be the key aspects of value presented by
	degree accreditation processes?\\
	
	\item [RQ2: ]What are perceived as the key challenges for future degree accreditation processes.\\
	
\end{description}

\section{Data collection and population}

\section{Method}

\section{Results}

\section{Discussion}

\section{Conclusions and future work}


Test citations~\cite{crick-et-al:fie2019,itnowcyber:2019,crick-et-al-accred:cep2020,itnowaccred:2020,crick-et-al:iticse2020,irons-et-alposter:sigcse2021}.

%\section*{Acknowledgment}

%\section*{References}
%\IEEEtriggeratref{38}

\bibliographystyle{IEEEtran}
%\bibliography{security}
\bibliography{IEEEabrv,FIE2022}


\end{document}
